\resheading{研究项目}
  \begin{itemize}[leftmargin=*]
    \item
      \ressubheading{美国五州阿片类毒品滥用状况的数据分析}{团队队长}{}{}
      {\small
      \begin{itemize}
        \item 对美国五个州各县的阿片类毒品的使用数据以及社会经济基本数据进行预处理,并数据可视化,得到此类毒品扩散的基本特征
        \item 随后分别使用聚类分析、灰色预测得到此类毒品扩散的地理特征和时间特征,并建立阈值分析模型,用于判断各县毒品泛滥情况
        \item 对各县社会经济基本数据进行主成分分析,得到用于判断各县基本社会经济情况的四项指标的分数,使用logistic回归,预测各县毒品泛滥情况,进行灵敏度分析,并提出针对性政策,定量分析其效用    
      \end{itemize}
      }
          \item
      \ressubheading{审视与拯救: 美国四州能源状况的评估与举措}{团队队长}{}{}
      {\small
      \begin{itemize}
        \item 利用美国国家统计局数据进行数据分析,得到各州能源概况,并建立评价指标,选择能源使用最为优良的州
        \item 对各州近几十年能源发展状况进行分析,使用时间序列模型预测未来数十年能源状况,并为其设立相应目标  
      \end{itemize}
      }
 %    \item
 %     \ressubheading{高温作业专用服装设计}{团队队长}{}{}
 %     {\small
 %     \begin{itemize}
 %       \item 以PDE方法研究高温环境下多层特殊材质隔热服装温度传导问题,求Robin与Dirichlet边值条件下数值解
 %       \item 使用服装隔热(clothing insulation)模型,利用多变量最优化问题研究服装最适厚度    
 %     \end{itemize}
 %     }
   % \item
   %   \ressubheading{大学生创新创业能力影响因素与发展机制研究  }{主要参与人}{}{}
   %   {\small
   %   \begin{itemize}
   %     \item 提出一种多维度创新创业能力影响因素模型,并采用定性与定量相结合的方式,归纳出影响创新创业能力的因素并进行比较,构建具有实操性的大学生创新创业能力发展机制。   
   %  \end{itemize}
   %   }
  \end{itemize}